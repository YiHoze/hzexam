
\documentclass{oblivoir}
\usepackage{fapapersize}
\usefapapersize{210mm,297mm,30mm,*,30mm,*}
\usepackage{hzexam}
\QAsetup{question-no=1, teacher=true, answer-sheet=false}
\setmainhangulfont{Noto Serif CJK KR}[BoldFont={* Bold}]
\begin{document}
\begin{premise}[3]
If there is an error, select the one underlined. Some sentences are correct and no sentence contains more than one error.
\end{premise}

\qa{
Allan \ulans{is} afraid \ulans{of} the rain\ulans{, he} \ulans{likes} the thunder.
}{
\ca{3}
}

\qa{
Maria, who had \ulans{just} eaten, thought \ulans{concerning} \ulans{having} a candy bar \ulans{or} ice cream. \ulans{No error}
}{
\ca{2}
}

\qa{
Last \ulans*{spring} the roofers replaced the asphalt shingles \ulans*{that} a heavy snowstorm \ulans*{damages} during the \ulans*{preceding} winter.
}{
\ca{3}
}\begin{premise}
빈 칸을 알맞은 말로 채우시오.
\end{premise}

\qa{
I used the \ulans{\phantom{pencil}} to draw a \ulans{\phantom{picture}} for my mom.
}{
\ca*{pencil, picture}
}

\qa{
Every night, we read a \uline{\phantom{book}} together in bed.
}{
\ca*{book}
}
\qa{
위대한 신 도깨비의 이름은 무엇인가요?
}[
김선 |
지은탁 ;
왕여 |
김신
]{
\ca{4}
}

\end{document}