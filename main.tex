%\documentclass[a4paper]{article}
\documentclass[a4paper]{oblivoir}
\usepackage{hyperref}
\usepackage{hzexam}


\title{hzexam.sty: 문제 은행 만들기}
\author{이호재}

\setmainhangulfont{Noto Serif CJK KR}[BoldFont={* Bold}]
\def\baselinestretch{1.25}
\setlength\parskip{1ex}
\NewDocumentEnvironment{example}{}{\color{violet}}{}

\begin{document}

\maketitle

\section{문항 만들기}

내가 어렸을 때 초등학교 교사였던 아버지는 종종 집에서 철필과 등사지를 이용하여 시험지를 작성하였다. 그때 그것이 얼마나 고단한 작업인지 나는 알지 못했고, 다만 나는 악필이라 선생님이 될 수 없으리라 생각했다.

복사기가 등장하면서 시험지 작성이 수월해졌다. 
중학교 때 한문 선생님이 오로지 모 출판사의 참고서를 시험 출제에 이용한다는 것을 한 친구가 알아차렸고, 우리는 앞다투어 그 참고서를 구입했다.
기말고사에서 우리는 평균 80 점이 넘는 놀라운 성적을 거두었으나, 그 다음 학기에는 곤두박질쳤다.

같은 일을 되풀이하는 것을 몹시 싫어하는지라, 선생님들이 수고를 덜기를 바라는 마음에, 시험 문항들을 데이터로서 다루는 몇 가지 매크로들을 만들었다. 

잠시 짬을 내어 문제 유형들을 살펴보았다. 
매우 다양한 것 같지만, 내가 아는 범위에서, 네 가지로 뭉뚱그릴 수 있다.

\begin{itemize}
\item 답이 몇 개이든, 보기 있는 객관식. 올바른 것끼리 연결하는 문제도 이 유형에 포함된다.
\item 답이 몇 개이든, 보기 없는 객관식. 주어진 문장에서 어법이나 문법에 맞지 않는 요소를 고르는 문제, 그리고 우리는 이용하지 않지만 답이 참 또는 거짓인 문제.
\item 단답형.
\item 서술형. 풀이 과정을 요구하는 수학 문제가 대표적이다.
\end{itemize}

\verb|\qa| 명령은 이 모든 유형들을 다룰 수 있다.
문제 유형에 따라 다른 매크로를 사용하는 것이 더 효율적인지 확신하지 못하여 하나로 만들었다.

\begin{boxedverbatim}
\qa{문제}[보기 | 보기 ; ...]{정답}
\end{boxedverbatim}

작성자 입장에서 다음과 같은 모양이 읽기에 편할 것이다.

\begin{boxedverbatim}
\qa{
위대한 신 도깨비의 이름은 무엇인가요?
}[ 
김선 | 
지은탁 ; 
왕여 | 
김신 
]{
\ca{4}
}
\end{boxedverbatim}

보기 구분자로 세로선(|) 또는 세미콜론(;)을 사용할 수 있다.

\begin{example}
\qa{
위대한 신 도깨비의 이름은 무엇인가요?
}[ 
김선 | 
지은탁 ; 
왕여 | 
김신 
]{
\ca{4}
}
\end{example}

정답이 보이는 까닭은 이 문서가 강사를 위한 것으로 설정되어 있기 때문이다.

\begin{boxedverbatim}
\QAsetup{teacher=true}
\end{boxedverbatim}

\verb|\ca| 명령은 객관식 정답을 \verb|\ca*| 명령은 단답형 정답을 나타내기 위한 것이다.
객관식 및 단답형 정답은 문제의 마지막 줄 끝에 온다.
밑줄이 붙은 \verb_\qa|_가 문제 유형이 서술형임을 가리킨다. 실제 하는 일은 답을 문제 옆이 아닌 그 아래 위치시키는 것에 불과하다.

\begin{boxedverbatim}
\qa|{
$y' = (y + 4x)^2$
}{
너무 어려워요.
선생님이 풀어 주세요.
}
\end{boxedverbatim}

\begin{example}
\qa|{
$y' = (y + 4x)^2$
}{
너무 어려워요.
선생님이 풀어 주세요.
}
\end{example}

문제 은행이 별것인가, 문항을 담은 많은 텍 파일들을 만들고 그것들 가운데 일부를 골라 삽입하는 것으로 충분하지 않은가?
하지만 파일 하나에 문항 하나를 담는 것은 곤란하다.
국어와 영어에서는 지문 하나에 두세 문항들이 따르는 것이 일반적이다.
수학에서는 비슷하게 문항 하나에 하위 문제들이 올 수 있다. 
그래서 \texttt{premise} 환경을 고안했다.

\begin{boxedverbatim}
\begin{premise}[2]
다음은 발표문이다. 물음에 답하시오.
\end{premise}

\begin{premise}*
다음 방정식을 푸시오. [각 10점]
\end{premise}
\end{boxedverbatim}

별표는 수학 문제를 위한 것이다. 문항 번호가 붙고, 그 아래 오는 \verb|\qa|에는 하위 번호들이 붙는다.

중괄호로 감싼 숫자는, 국어나 영어를 위한 것인데, 그 지문에 딸린 문항의 수를 가리킨다.
예를 들어, \texttt{[3]}으로 지정하면 다음 문항 번호가 7일 때 \textbf{[7〜9]}가 표시된다.

문항 파일들을 작성하여 삽입하는 것은 다소 성가시다.
\texttt{QAwrite} 환경은 그 내용을 개별 파일에 저장한다.

\begin{boxedverbatim}
\begin{QAwrite}
\begin{premise}
... 
\end{premise}
\qa{...}[...]{...}
\qa{...}{...}
\end{QAwrite}
\end{boxedverbatim}

파일 이름은 다음과 같은 설정을 따른다.

\begin{boxedverbatim}
\QAsetup{file=eng/grammar, counter=5, zeros=3, overwrite=true}
\QAsetup{file=한국사/고려, counter=10, zeros=3, overwrite=true}
\end{boxedverbatim}

이 설정에 따라 grammar-005.tex, grammar-006.tex, \ldots\ 파일이 eng 폴더에 저장된다.
물론 eng 폴더는 미리 만들어져 있어야 한다.

\texttt{counter} 설정에 주의를 기울이지 않으면 이미 다른 내용으로 저장된 파일을 덮어쓸 수 있다.
이를 예방하기 위해 \texttt{overwrite} 옵션이 있지만, 큰 도움이 되지 않을 것이다.
내용을 처음 작성할 때는 거듭 수정해야 하므로 불가피하게 \texttt{true}로 설정해야 할 것이다.
디폴트도 true이다.

\section{시험지 만들기}

문항 파일들을 충분히 만들었다면, 그것들을 삽입하여 시험지를 만들어 보자.

\begin{boxedverbatim}
\QAinput{eng/grammar}{2, 5, 8-10, 15}
\QAinput{eng/usage}{3, 4, 27-32, 112}
\QAinput{}{voca*}
\end{boxedverbatim}

eng 폴더에서 grammar-002.tex 파일을 비롯하여 지정된 번호들과 일치하는 파일들이 삽입될 것이다.
마지막 방법은 현재 폴더에서 파일 이름이 voca로 시작하는 모든 파일들을 삽입하라는 뜻이다.
이 방법은 \texttt{-shell-escape} 컴파일 옵션을 요구한다.
두 방법을 혼용하는 것은 가능하지 않다.

\section{문제집 만들기}

문제집에 정답지가 없다면, 학원 강사가 자신의 수업을 위해 만든 게 아니라면, 잘 팔릴 리 없다.

\begin{boxedverbatim}
\QAsetup{answr-sheet=true}
\InputAnswersSheet
\end{boxedverbatim}

\texttt{answer-sheet} 옵션이 \texttt{true}로 설정되면 \verb|\qa| 명령이 정답지에 정답을 하나씩 추가한다.
메인 파일이 \texttt{foo.tex}이라면 정답지 파일은 \verb|foo_answers_sheet.tex|이 된다.

\verb|\InputAnswersSheet| 명령이 정답지 파일을 삽입한다.

\section{결어}

판형이나 잡다한 장식과 관련된 것은 작성자가 궁리할 일이어서 고려하지 않았다.
그러나 만들다 보니 다른 이러저러한 문제들이 보인다.
이를테면, 정답지가 아니라 풍부한 해설이 포함된 해답지를 만들고 싶다면? 
여차저차하게 후크를 추가하여 매크로를 고치면 될 것이다.
폰트 따위를 설정 명령을 통해 변경하게 만드는 것도 그다지 어렵지 않을 것이다.
단지 성가실 뿐.

다음 단계로 생각해볼 것은 문항 파일들을 데이터베이스에 넣는 것이다.
개인이 아니라 다수가 집단으로 문항들을 만들고 관리하려면 데이터베이스가 필수적이다.
그것은 텍으로 구현할 것이 아니고 프로그래밍 언어에 맡겨야 한다.
웹 브라우저에서 이런저런 조건들을 설정하면 시험지 하나가 다운로드되는 그런 것 \ldots\ 만들려면 골치 꽤나 아프겠다.

\section{빈약한 예들}

\href{https://www.suneung.re.kr/boardCnts/list.do?boardID=1500234&m=0403&s=suneung}{대학수학능력 기출 문제}를 베껴 예로 사용할까 하다가 깊이 고민하지 않고 포기했다. 
재미로 할 만한 짓이 못 된다.

\begin{boxedverbatim}
\QAsetup{file=subject, counter=1, question-no=1}
\end{boxedverbatim}
\QAsetup{file=subject, counter=1, question-no=1}

\begin{QAwrite}

\begin{premise}[3]
If there is an error, select the one underlined. Some sentences are correct and no sentence contains more than one error. 
\end{premise}

\qa{
Allan \ulans{is} afraid \ulans{of} the rain\ulans{, he} \ulans{likes} the thunder. 
}{
\ca{3}
}

\qa{
Maria, who had \ulans{just} eaten, thought \ulans{concerning} \ulans{having} a candy bar \ulans{or} ice cream. \ulans{No error}
}{
\ca{2}
}

\qa{
Last \ulans*{spring} the roofers replaced the asphalt shingles \ulans*{that} a heavy snowstorm \ulans*{damages} during the \ulans*{preceding} winter. 
}{
\ca{3}
}

\end{QAwrite}

\begin{QAwrite}

\begin{premise}
빈 칸을 알맞은 말로 채우시오.
\end{premise}

\qa{
I used the \ulans{\phantom{pencil}} to draw a \ulans{\phantom{picture}} for my mom.
}{
\ca*{pencil, picture}
}

\qa{
Every night, we read a \uline{\phantom{book}} together in bed.
}{
\ca*{book}
}

\end{QAwrite}


\begin{QAwrite}

\qa{
위대한 신 도깨비의 이름은 무엇인가요?
}[ 
김선 | 
지은탁 ; 
왕여 | 
김신 
]{
\ca{4}
}

\end{QAwrite}

\begin{QAwrite}
\begin{premise}*
다음의 1계 미분방정식을 푸시오. [각 5점]
\end{premise}

\qa*|{
$y' = (y + 4x)^2$
}{
너무 어려워요.
선생님이 풀어 주세요.
}

\qa*|{
$3x^2y^2y' + x(x - 1)y^3 = x^3e^x$
}{
이것도 어려워요.
집에 가서 엄마한테 물어볼게요.
}
\end{QAwrite}

\section{정답지}
\InputAnswersSheet

\end{document}